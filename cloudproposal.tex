\documentclass{article}
\title{Migration to the Cloud}
\author{Team 5}
\begin{document}
\maketitle
\par Just as any global bank, real or not, would experience, a vast amount of extremely sensitive fiscal data must be sifted through constantly and at great rates. For this reason, among others, We propose to move to the "fully cloud native" approach to cloud migration. We detail our reasoning below:

\section{Scalability \& Server Hammering}
\par Currently, our bank is being run off a local machine, often a labtop (the creative reader may be able to suspend their disbelief and consider it as a server). At some point, the number of transactons in the system will exceed the storage space of that laptop (or server).

\par Similarly, the variable rate of transactions could possibly cause some asynchronous problems relating to reading/ writing to databases, which may lead to very unhappy customers.

\par Outsourcing database and server concerns to a cloud provider mitigate these risks. It allows the dynamic allocation of server overhead at almost instantaneous speed.

\section{Endurance \& Quick Deployment}
\par Our banking app in its current state is only active as long as the local machine is active. This, by itsself is an extremely lousy and shortsighted approach for a worldwide service. A single hardware failure, power outtage, system crash an cause the entire service to grind to a halt. 
\par This, obviously, is not acceptable for a banking service. By migrating server processing to a cloud provider, we can eliminate this vulnerability and ensure continuous availability.


\section{Security}
\par Cloud providers offer a large 

\section{Telemetry}

\section{Load Balancing}



\end{document}

